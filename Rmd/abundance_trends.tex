% Options for packages loaded elsewhere
\PassOptionsToPackage{unicode}{hyperref}
\PassOptionsToPackage{hyphens}{url}
%
\documentclass[
]{article}
\usepackage{lmodern}
\usepackage{amsmath}
\usepackage{ifxetex,ifluatex}
\ifnum 0\ifxetex 1\fi\ifluatex 1\fi=0 % if pdftex
  \usepackage[T1]{fontenc}
  \usepackage[utf8]{inputenc}
  \usepackage{textcomp} % provide euro and other symbols
  \usepackage{amssymb}
\else % if luatex or xetex
  \usepackage{unicode-math}
  \defaultfontfeatures{Scale=MatchLowercase}
  \defaultfontfeatures[\rmfamily]{Ligatures=TeX,Scale=1}
\fi
% Use upquote if available, for straight quotes in verbatim environments
\IfFileExists{upquote.sty}{\usepackage{upquote}}{}
\IfFileExists{microtype.sty}{% use microtype if available
  \usepackage[]{microtype}
  \UseMicrotypeSet[protrusion]{basicmath} % disable protrusion for tt fonts
}{}
\makeatletter
\@ifundefined{KOMAClassName}{% if non-KOMA class
  \IfFileExists{parskip.sty}{%
    \usepackage{parskip}
  }{% else
    \setlength{\parindent}{0pt}
    \setlength{\parskip}{6pt plus 2pt minus 1pt}}
}{% if KOMA class
  \KOMAoptions{parskip=half}}
\makeatother
\usepackage{xcolor}
\IfFileExists{xurl.sty}{\usepackage{xurl}}{} % add URL line breaks if available
\IfFileExists{bookmark.sty}{\usepackage{bookmark}}{\usepackage{hyperref}}
\hypersetup{
  pdftitle={Trends in Seasonal Abundance},
  hidelinks,
  pdfcreator={LaTeX via pandoc}}
\urlstyle{same} % disable monospaced font for URLs
\usepackage[margin=1in]{geometry}
\usepackage{graphicx}
\makeatletter
\def\maxwidth{\ifdim\Gin@nat@width>\linewidth\linewidth\else\Gin@nat@width\fi}
\def\maxheight{\ifdim\Gin@nat@height>\textheight\textheight\else\Gin@nat@height\fi}
\makeatother
% Scale images if necessary, so that they will not overflow the page
% margins by default, and it is still possible to overwrite the defaults
% using explicit options in \includegraphics[width, height, ...]{}
\setkeys{Gin}{width=\maxwidth,height=\maxheight,keepaspectratio}
% Set default figure placement to htbp
\makeatletter
\def\fps@figure{htbp}
\makeatother
\setlength{\emergencystretch}{3em} % prevent overfull lines
\providecommand{\tightlist}{%
  \setlength{\itemsep}{0pt}\setlength{\parskip}{0pt}}
\setcounter{secnumdepth}{-\maxdimen} % remove section numbering
\ifluatex
  \usepackage{selnolig}  % disable illegal ligatures
\fi

\title{Trends in Seasonal Abundance}
\author{}
\date{\vspace{-2.5em}}

\begin{document}
\maketitle

Here we use catch and effort data from recreational fisheries in Juan de
Fuca Strait to estimate changes in total Chinook salmon abundance, while
accounting for seasonal patterns. By pairing these estimates with stock
composition from genetic sampling (2014-2019) we can also estimate
stock-specific abundance. To begin a few clarifications.

First, abundance is estimated assuming a fixed effort (overall mean of
monthly in all rec fisheries; \textasciitilde1350 boat days) applied for
each month and statistical area. Estimates for a region are then summed
across all stat areas in that region (i.e.~121 and 21 here). This
effectively turns fisheries dependent data into an index of abundance.
The predictions shown here are generated with a model containing two
components (one for abundance, one for composition). Each models monthly
changes with a seasonal smoother (analogous to LOESS), while estimating
interannual variation as a random effect. Happy to discuss details
further as needed.

Second, the Swiftsure foraging area corresponds to stat areas 121 and
21, while the portion of Haro that is in Canadian waters is 19B and 19C.
Unfortunately stratifying GSI samples by stat area (not to mention
sub-stat area) reduces the sample size to such an extent that I don't
trust the model predictions (mean of \textless10 samples per
month/year/area). For now I've estimated abundance at the stat area
level for areas 121/21 and applied stock composition estimates for Juan
de Fuca as a whole (defined here as areas 21, 121 and 20), the rationale
being that many of the Swiftsure fish will be migrating through Juan de
Fuca in the near future so stock composition should be correlated. We
can do something similar for area 19 in the southern Strait of Georgia,
but I'm not sure which areas should be rolled up and have held off for
now. Also makes it easier to interpret figures.

Third, creel data are limited to June-September for stat areas 121 and
21 so it's not possible to estimate abundance earlier/later in the year.
For the southern Strait of Georgia it will be May-September. Note that I
also haven't pulled new data from Wilf's files yet, which should include
2020 and, hopfully soon, 2021. There are also options if we want to
extend the stock-specific estimates back. Basically we can generate an
estimate of stock composition excluding annual effects (i.e.~composition
in an annual year) then generate an estimate for previous years by
sampling from the distribution of annual variability. These estimates
will have inflated uncertainty, but be statistically defensible.

Fourth, the stock groups I've chosen here are somewhat arbitrary. For
example, Fraser early includes both threatened yearling CUs
(spring/summer 4.2s and 5.2s), as well as highly abundant summer 4.1s.
The stock groups can be aggregated or split as needed, though generating
a large number of stocks will probably give the model some trouble
(ratio of parameters to data decreases). At least for Juan de Fuca it
probably makes sense to split out the different early run Fraser and
pool some of the less commonly encountered stocks (e.g.~Columbia,
coastal BC and coastal US). Trick will be finding mix of stock groupings
that is intuitive for both SoG and JdF.

All this is to say that this is a first pass and that now that the model
is built, it will be relatively easy to pass it different data
configurations to focus on specific hypotheses.

The following plot shows summed total abundance on the z-axis for
available months (x) and years (y). Abundance appears to have increased,
particularly early in the year.

\includegraphics{abundance_trends_files/figure-latex/agg_ridge-1.pdf}

These estimates can also be better visualized with estimates of
uncertainty (ribbon represents 95\% CI).

\includegraphics{abundance_trends_files/figure-latex/agg_ribbon-1.pdf}

The interannual trend can also be emphasizedwithout the seasonal smooths
as month specific estimates (interval represents 95\% CI).

\includegraphics{abundance_trends_files/figure-latex/agg_dot-1.pdf}

Slightly different patterns emerge when considering stock-specific
abundance. Note change in color scale because fewer years available.

\includegraphics{abundance_trends_files/figure-latex/stock_ridge-1.pdf}

Note here that the y-axis varies among stock groups. Here it starts to
become clear that increases in the abundance of Fraser Early (likely
summer 4.1s, i.e.~South Thompson) and Puget Sound are driving increases
in total abundance in these stat areas.

\includegraphics{abundance_trends_files/figure-latex/stock_ribbon-1.pdf}

\includegraphics{abundance_trends_files/figure-latex/stock_dot-1.pdf}

\end{document}
