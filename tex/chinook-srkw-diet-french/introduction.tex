Les eaux canadiennes du Pacifique abritent trois écotypes d'épaulards (\textit{Orcinus orca}), chacun génétiquement distinct et présentant des différences significatives en morphologie, comportement et alimentation. L'écotype résident piscivore comprend deux populations — l'épaulard résident du Nord (ÉREN) et l'épaulard résident du Sud (ÉRES). Le saumon du Pacifique (\textit{Oncorhynchus} spp.) forme une grande partie de leur alimentation, le saumon chinook (\textit{O. tshawytscha}) étant identifié comme l'espèce proie dominante \citep{fordSelectiveForagingFisheating2006, fordDietarySpecializationTwo1998, hansonSpeciesStockIdentification2010}. De 1962 à 1973, les deux populations résidentes ont subi des prélèvements importants en raison de la pêche de capture vivante \citep{biggLivecaptureKillerWhale1975}. Un recensement annuel a lieu depuis la fermeture de la pêcherie, les données récentes indiquant que la population d'ÉREN continue de présenter une augmentation annuelle de 2,5 à 3,5\% et compte approximativement 341 individus \citep{dfoPopulationStatusUpdate2023}. La population d'ÉRES a échoué à maintenir une trajectoire de croissance positive et ne consiste qu'en 74 individus. Les populations sont inscrites en vertu de la \textit{Loi sur les espèces en péril} comme menacées et en voie de disparition, respectivement \citep{cosewicCOSEWICAssessmentUpdate2001, cosewicCOSEWICAssessmentStatus2023}.

La distribution de l'ÉRES s'étend de la Californie centrale au sud-est de l'Alaska \citep{thorntonSouthernResidentKiller2022}. Pendant les mois d'été, l'ÉRES utilisait historiquement de manière extensive la mer des Salish, qui comprend le détroit de Georgie, le détroit d'Haro, Puget Sound, et le détroit de Juan de Fuca \citep{fordSelectiveForagingFisheating2006, fordHabitatsSpecialImportance2017}. Plus récemment, les zones à l'ouest du détroit de Juan de Fuca, y compris le banc de La Perouse, le banc Swiftsure, et les canyons sous-marins proximaux ont été identifiés comme habitat essentiel pendant l'été ainsi qu'en hiver \citep{fordHabitatsSpecialImportance2017, thorntonSouthernResidentKiller2022}. L'utilisation de l'habitat par l'ÉRES est associée aux migrations du saumon du Pacifique, et en particulier du saumon chinook, qui se trouvent dans la mer des Salish et les eaux associées du plateau continental toute l'année \citep{oneillMarineDistributionLife2009, chamberlinEffectsNatalOrigin2014, freshwaterIntegratedModelSeasonal2021}.

Un des plusieurs facteurs hypothétiques des faibles taux de croissance de la population chez l'ÉRES est la disponibilité réduite des proies, avec un accent particulier sur les déclins de l'abondance du saumon chinook. Les observations des régimes alimentaires de l'ÉRES sont typiquement dominées par le saumon chinook, bien que la diversité des proies augmente en hiver pour inclure le saumon coho (\textit{O. kisutch}), le saumon kéta (\textit{O. keta}), et la morue charbonnière (\textit{Anoplopoma fimbria}) \citep{fordLinkingKillerWhale2010, fordEstimationKillerWhale2016, hansonEndangeredPredatorsEndangered2021}. De plus, les taux démographiques de l'ÉRES covariaient avec les indices à grande échelle de l'abondance du saumon chinook \citep{wardIncreasedPopulationDensity2009, fordLinkingKillerWhale2010}, bien que ces effets se soient avérés difficiles à estimer avec précision étant donné la longue durée de vie des épaulards \citep{velez-espinoRelativeImportanceChinook2015}. Les indices de l'abondance du saumon chinook étaient également corrélés avec les changements de condition individuelle de l'ÉRES \citep{stewartSurvivalFattestLinking2021}, les changements de comportement d'alimentation \citep{holtEffectsVesselDistance2021}, l'utilisation des habitats de la mer des Salish \citep{ettingerShiftingPhenologyEndangered2022, stewartTraditionalSummerHabitat2023}, et, via des modèles d'écosystème équilibrés en masse, les déficits caloriques dans la population d'ÉRES \citep{coutureRequirementsAvailabilityPrey2022}. Néanmoins, certaines preuves suggèrent que les relations entre les taux démographiques de l'ÉRES et l'abondance du chinook se sont affaiblies dans le temps, peut-être en raison de la compétition avec les épaulards résidents du Nord, qui ont augmenté en abondance \citep{nelsonIdentifyingDriversDemographic2024}, ou de l'effet accru d'autres impacts anthropiques sur la santé individuelle (p. ex., mobilisation de contaminants à partir des réserves lipidiques ; facteurs de stress physiologiques tels que les perturbations physiques et acoustiques).

Évaluer les besoins en proies de l'ÉRES, et adapter les interventions de gestion pour répondre à ces besoins, est difficile étant donné la diversité écologique du saumon chinook. Le saumon chinook, comme la plupart des saumons du Pacifique, est anadrome et sémelpare, mais diffère de ses congénères de plusieurs façons notables. Le saumon chinook a la distribution la plus étendue des saumons du Pacifique dans le Pacifique oriental, s'étendant du fleuve Yukon à la Californie centrale. Il présente des stratégies d'histoire de vie diverses pendant l'élevage en eau douce (moins d'un à plusieurs années) et marin (moins d'un à plus de cinq années). En raison de leur âge moyen relativement âgé à la maturité et de leur régime marin principalement piscivore, le saumon chinook atteint la plus grande taille corporelle des saumons du Pacifique et peut dépasser 15 kilogrammes \citep{healeyLifeHistoryChinook1991}.

Le saumon chinook montre une forte fidélité au site de frai et au calendrier de migration, résultant en de nombreuses populations génétiquement distinctes \citep{healeyLifeHistoryChinook1991, quinnBehaviourEcologyPacific2018}. Les populations de saumon chinook peuvent différer selon plusieurs dimensions écologiques incluant la stratégie d'histoire de vie juvénile (c.-à-d., entrant dans l'océan avec (yearling) ou sans (subyearling) hivernage comme juvéniles en eau douce), la phénologie de migration en eau douce des adultes (c.-à-d., calendrier de remontée printanière, estivale, ou automnale), et la distribution marine. Contrairement à de nombreuses espèces de saumons du Pacifique, un nombre substantiel de populations de saumon chinook, principalement des individus subyearling de remontée automnale, passent toute leur vie marine dans les eaux côtières relativement proches de leurs lieux d'entrée océanique \citep{oneillMarineDistributionLife2009, chamberlinEffectsNatalOrigin2014, freshwaterIntegratedModelSeasonal2021}. D'autres populations de remontée estivale et automnale, ainsi que certaines populations yearling, migrent vers le nord sur des distances considérables, mais demeurent sur le plateau continental \citep{weitkampMarineDistributionsChinook2010, ctc2021AnnualReport2022}. Les populations yearling de remontée printanière qui frayent dans les parties intérieures de grands bassins versants se dispersent typiquement au large \citep{weitkampMarineDistributionsChinook2010, ctc2021AnnualReport2022}.

De nombreux stocks de saumon chinook (utilisé tout au long pour faire référence à une ou plusieurs populations de frai distinctes) ont décliné en abondance au cours des 50 dernières années coïncidant avec la surpêche, la perte d'habitat, et les changements de productivité. Dans toute la Californie, les déclins de l'habitat d'eau douce et les conditions environnementales changeantes ont résulté en des réductions substantielles de la diversité des histoires de vie et des pêcheries moins résilientes \citep{munschOneHundredseventyYears2022}. Dans le bassin versant du fleuve Columbia, cinq des sept unités évolutionnaires significatives (UES) sont inscrites comme menacées ou en voie de disparition sous l'Endangered Species Act \citep{nmfsEndangeredSpeciesAct2020}. L'UES du saumon chinook de Puget Sound demeure également inscrite comme menacée \citep{fordStatusReviewUpdate2011}. Dans le sud de la Colombie-Britannique, la majorité des unités de conservation (UC) de saumon chinook évaluées ont été identifiées comme menacées ou en voie de disparition \citep{cosewicChinookSalmonOncorhynchus2018}. Notamment, les déclins se sont produits malgré des investissements considérables dans la remédiation de l'habitat \citep{jaegerReturnsInvestmentRestoration2023} et des réductions des taux d'exploitation \citep{ctc2021AnnualReport2022}. L'étendue spatiale large des déclins, incluant dans les régions avec des habitats d'eau douce relativement vierges, a mené à un consensus scientifique croissant que la faible survie marine et les réductions de l'âge à la maturité ont réduit la productivité et contraint le rétablissement \citep{dornerSpatialTemporalPatterns2017, ruffSalishSeaChinook2017, ohlbergerDemographicChangesChinook2018, welchSynthesisCoastwideDecline2021, freshwaterNonstationaryPatternsDemographic2022}.

Néanmoins, les déclins de l'abondance du saumon chinook ne sont pas uniformes. Les déclins de l'abondance des stocks yearling de remontée printanière dans les grands bassins versants ont été particulièrement sévères, mais plusieurs stocks subyearling de remontée estivale qui frayent dans l'intérieur des mêmes bassins versants ont augmenté en abondance \citep{cosewicChinookSalmonOncorhynchus2018, ctc2021AnnualReport2022, atlasTrendsChinookSalmon2023}. Les tendances de l'abondance des stocks côtiers de remontée automnale sont mitigées avec certains stocks, même dans la même région géographique, augmentant tandis que d'autres déclinent \citep{ctc2021AnnualReport2022, atlasTrendsChinookSalmon2023}. En raison de ces patrons divergents, l'abondance du saumon chinook peut être relativement stable, en augmentation, ou en diminution selon les stocks considérés. Par exemple, puisque les stocks subyearling de remontée estivale et automnale sont abondants relativement aux stocks de remontée printanière, spécialement étant donné la production substantielle d'écloserie, l'abondance agrégée du saumon chinook dans les zones marines s'étendant du sud-est de l'Alaska à l'île de Vancouver est demeurée relativement stable au cours des 30 dernières années \citep{ctc2021AnnualReport2022}. La variabilité parmi les stocks de saumon chinook dans les tendances d'abondance, avec la puissance statistique modeste, peut contribuer aux relations incertaines ou s'affaiblissant entre les estimations grossières de l'abondance du saumon chinook et les taux démographiques de l'ÉRES \citep{nelsonIdentifyingDriversDemographic2024}.

Toutes choses étant égales par ailleurs, les stocks de saumon chinook plus abondants devraient être des composantes plus prévalentes du régime alimentaire de l'ÉRES. Cependant, la contribution de différents stocks de saumon chinook à la base de proies de l'ÉRES peut covarier avec les caractéristiques écologiques. Premièrement, le calendrier de migration du saumon chinook déterminera quand des stocks de saumon spécifiques sont abondants dans les régions, telles que la partie ouest du détroit de Juan de Fuca, où l'ÉRES s'alimente de manière disproportionnée \citep{thorntonSouthernResidentKiller2022}. L'ÉRES montre des patrons saisonniers de condition, indiquant qu'il peut éprouver des limitations de proies avant les observations en mai (Fearnbach et al. 2020; J. Durban, comm. pers.). Une diminution générale de la condition corporelle pendant les mois d'hiver suggère que les stocks de saumon chinook de remontée précoce (c.-à-d., de remontée printanière et estivale précoce) peuvent fournir une ressource critique. En effet, l'utilisation des habitats de la mer des Salish par l'ÉRES s'est produite plus tard dans l'année coïncidant avec les déclins de l'abondance du saumon chinook de remontée printanière du fleuve Fraser et les augmentations de l'abondance des stocks avec un calendrier de migration plus tardif \citep{ettingerShiftingPhenologyEndangered2022}. Deuxièmement, les distributions marines spécifiques aux stocks interagissent avec le calendrier de remontée pour déterminer l'étendue à laquelle des stocks spécifiques de saumon chinook se chevauchent spatialement et temporellement avec l'ÉRES. Alors que de nombreux stocks de saumon chinook ne sont présents dans les zones côtières du sud de la Colombie-Britannique qu'immédiatement avant les migrations de frai, d'autres sont résidents pour toute leur histoire de vie marine et sont disponibles comme proies pour l'ÉRES toute l'année \citep{oneillMarineDistributionLife2009, chamberlinEffectsNatalOrigin2014, freshwaterIntegratedModelSeasonal2021}. Troisièmement, les stocks de saumon chinook diffèrent en âge à la maturité \citep{healeyLifeHistoryChinook1991, ohlbergerDemographicChangesChinook2018}, taille selon l'âge \citep{ohlbergerDemographicChangesChinook2018, xuClimateEffectsSize2020}, et contenu lipidique \citep{oneillEnergyContentPacific2014, lernerSeasonalVariationLipid2023, freshwaterSeasonalVariabilityCondition2024}, qui déterminent collectivement leur valeur énergétique relative. Il est possible que l'ÉRES sélectionne des éléments de proie plus grands ou plus riches en énergie de manière disproportionnée à leur abondance. En effet, les individus de saumon chinook plus grands et plus âgés semblent être plus communs dans les régimes alimentaires des épaulards résidents que dans les pêcheries proximales \citep{fordSelectiveForagingFisheating2006}.

Déterminer la valeur relative des stocks de saumon chinook pour les ÉRES fournit un aperçu des emplacements et des moments où l'ÉRES peut être le plus sensible aux réductions de la disponibilité des proies. Un sous-ensemble de stocks de saumon chinook « de haute valeur » peut fournir un signal écologique plus fort reliant les traits démographiques de l'ÉRES et la condition individuelle à l'abondance des proies. Évaluer la valeur relative des stocks de saumon chinook permet également aux gestionnaires de ressources de prioriser les interventions de gestion qui sont les plus susceptibles d'améliorer la disponibilité des stocks de haute valeur. Alors que la recherche précédente a déterminé que la contribution relative des stocks de saumon chinook aux régimes alimentaires de l'ÉRES varie saisonnièrement \citep{hansonSpeciesStockIdentification2010, hansonEndangeredPredatorsEndangered2021}, les données de composition des proies n'ont pas été directement comparées au champ de proies de fond.

Ici, nous utilisons de multiples lignes de preuves pour évaluer la contribution relative de différents stocks et classes de taille de saumon chinook aux régimes alimentaires de l'ÉRES dans le sud de la Colombie-Britannique. Nous notons qu'en raison des limitations de données, notre analyse se concentre sur les interactions du saumon chinook de l'ouest du détroit de Juan de Fuca au sud du détroit de Georgie pendant les mois d'été (mai-octobre) et n'est pas une évaluation complète des besoins en proies. Néanmoins, la région a été identifiée comme habitat essentiel pour l'ÉRES \citep{dfoIdentificationHabitatsSpecial2017} et la condition des individus ÉRES est corrélée avec les opportunités d'alimentation dans ces zones \citep{ettingerShiftingPhenologyEndangered2022, stewartSurvivalFattestLinking2021, stewartTraditionalSummerHabitat2023}. Premièrement, nous avons utilisé des échantillons de débris de proies pour évaluer la composition des stocks et de la taille des événements d'alimentation de l'ÉRES dans différentes parties de l'habitat essentiel de l'ÉRES. Deuxièmement, nous avons utilisé des échantillons collectés des pêcheries récréatives comme un indice de la disponibilité relative de différents stocks et classes de taille de saumon chinook pour l'ÉRES. Nous avons utilisé des modèles spatio-temporels ajustés aux données dépendantes de la pêcherie pour prédire la composition des stocks et la composition de la taille à des intervalles hebdomadaires à travers l'habitat essentiel de l'ÉRES. Troisièmement, nous avons conduit une analyse de simulation pour déterminer si les échantillons de proies observés déviaient des prédictions dérivées du modèle pour tester l'évidence de la sélection par l'ÉRES de stocks ou de classes de taille de manière disproportionnée à leur abondance. Quatrièmement, nous avons colligé les données d'abondance terminale (c.-à-d., l'abondance du saumon chinook mature énuméré immédiatement avant ou après l'entrée en eau douce) pour comparer les patrons d'abondance agrégée versus spécifique aux stocks dans l'habitat essentiel canadien. Finalement, nous avons intégré ces résultats avec de l'information sur les différences spécifiques aux stocks en condition du saumon chinook, contribution d'écloserie, composition d'âge, et comportement pour identifier les interventions de gestion potentielles pour augmenter la disponibilité des proies de l'ÉRES.