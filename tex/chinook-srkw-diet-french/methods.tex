\subsection{Modèles statistiques}

\subsubsection{Composition des stocks}

Les données de composition des stocks représentent la proportion d'un échantillon total appartenant à chaque stock. Puisque, quand une MTI n'est pas présente, l'identité du stock est estimée basée sur les marqueurs moléculaires et les modèles de mélange, l'inférence basée sur les données de composition des stocks devrait tenir compte d'au moins deux sources d'incertitude — la variabilité associée à la taille d'échantillon d'une observation et la variabilité associée à la précision de l'assignation de stock. Les statistiques sommaires, telles que l'abondance relative moyenne d'un stock, échouent souvent à tenir compte des différences de taille d'échantillon parmi les événements d'échantillonnage, ainsi que l'autocorrélation parmi les événements d'échantillonnage. Par exemple, les données de composition des stocks collectées dans des strates spatiales adjacentes sont peu susceptibles d'être statistiquement indépendantes. De manière similaire, l'incertitude dans les probabilités d'assignation de stock génétique est souvent ignorée en utilisant des seuils arbitraires (p. ex., 75\% de probabilité) pour assigner un individu à un stock avec une confiance absolue.

Pour adresser ces problèmes, nous avons utilisé des modèles statistiques pour évaluer les différences de l'abondance relative des stocks parmi les emplacements et les mois, tenir compte de la variation de l'intensité d'échantillonnage, et propager l'incertitude associée aux assignations de stock individuelles. Les modèles linéaires généralisés (MLG) et les modèles additifs généralisés (MAG) multinomiaux de Dirichlet sont de plus en plus utilisés pour tirer des inférences sur les données de proportions provenant de mesures continues \citep{doumaAnalysingContinuousProportions2019}, telles que les probabilités d'assignation de stock individuelles \citep{freshwaterIntegratedModelSeasonal2021, jensenIntroducingZoidMixture2022}. Cependant, la distribution multinomiale de Dirichlet ne peut pas facilement tenir compte des observations zéro \citep{doumaAnalysingContinuousProportions2019, jensenIntroducingZoidMixture2022}, qui sont communes dans les données de composition avec des événements d'échantillonnage relativement petits et des groupes rares, et les paramètres sous-jacents sont difficiles à interpréter. De plus, il n'y a actuellement aucun MLG ou MAG multinomial de Dirichlet spatialement explicite. Comme alternative, nous avons utilisé des MAG Tweedie multivariés pour quantifier la variabilité dans la composition des stocks à travers le temps et l'espace. Brièvement, le Tweedie multivarié assume que les données de composition proviennent d'un processus de Poisson éclairci et marqué qui peut être exprimé comme une distribution Tweedie \citep{fosterPoissonGammaModel2013, thorsonDietAnalysisUsing2022}. Les études de simulation suggèrent que le Tweedie multivarié a une performance similaire ou supérieure au multinomial de Dirichlet lors de l'analyse des données de composition \citep{thorsonMultivariateTweedieSelfweightingLikelihood2023}.

Nous avons tenté d'ajuster diverses structures de modèles aux données de restes de proies de l'ÉRES, mais il y avait trop peu d'observations pour capturer adéquatement la variabilité parmi les saisons, les strates spatiales, et les périodes d'échantillonnage précoce et actuelle. Ainsi, nous évaluons seulement qualitativement les patrons dans la composition des stocks et de la taille pour ces données.

Nous avons ajusté des MAG Tweedie multivariés aux données de saumon chinook dépendantes de la pêcherie pour quantifier la variabilité saisonnière et spatiale dans la composition des stocks, tout en tenant compte de la variabilité interannuelle de l'abondance. Nous avons défini un événement d'échantillonnage $j$, représentant tous les poissons individuels échantillonnés pour l'identité de stock dans un emplacement de pêche, une semaine, et une année donnés. Puisque les probabilités d'assignation ont été identifiées à l'échelle des populations de frai (c.-à-d., sous-stock), nous avons agrégé les données GSI en stocks (Tableau S1), en sommant, dans $j$, toutes les probabilités associées aux populations appartenant au stock $s$. Ainsi, la densité observée de $s$ dans un événement d'échantillonnage $d_{j_s}$ pouvait prendre toute valeur positive continue entre zéro (tous les individus échantillonnés avaient une probabilité zéro d'être assignés à $s$) à la taille totale de l'échantillon $n_j$ (tous les individus ont été assignés à $s$ avec 100\% de probabilité). Puisque les individus avaient souvent une probabilité non-zéro d'appartenir à de multiples stocks, $d_{j_s}$ n'était souvent pas un entier. Quand l'identité de stock était inférée d'une MTI, PBT, ou marque thermique d'otolithe, alors la probabilité d'assignation pour cet individu était de un. Puisque $d$ était une fonction de $n_j$, la précision des estimations de composition moyenne était mise à l'échelle avec le nombre d'individus inclus dans un événement d'échantillonnage.

Nous avons estimé les changements de l'abondance relative de chaque stock de saumon chinook en utilisant une distribution Tweedie avec moyenne $\mu$, puissance $p$, échelle $\phi$, et un lien logarithmique :

\begin{align}
\label{eq:comp_eq}
\tag{1}
d_{j_s} &\sim \operatorname{Tweedie} \left( \mu_{j_s}, p, \phi \right),\\
\tag{2}
\log(\mu_{j_s}) &= \alpha_s + \alpha_{y_s} + f_{w_{s}}w + f_{{g_s},{h_s}}({g,h})\\
\notag
\end{align}
\noindent

Où $\alpha_s$ est une ordonnée à l'origine représentant l'abondance moyenne de chaque stock, $\alpha_{y_s}$ est une ordonnée à l'origine aléatoire représentant les changements spécifiques à l'année $y$ de l'abondance du stock $s$, $f_{w_s}$ est une fonction lisse de processus gaussien pour la semaine d'échantillonnage $w$ qui varie parmi les stocks, et $f_{g_s,h_s}$ est une fonction lisse de spline de Duchon pour l'emplacement d'échantillon (abscisse $g$ et ordonnée $h$) qui varie parmi les stocks. Nous avons modélisé $\alpha_{y_s}$ en assumant une hyperdistribution avec moyenne zéro et écart-type $\sigma_y$. Les fonctions lisses ont tenu compte de la variation saisonnière dans la composition des stocks, tandis que les fonctions lisses de spline de Duchon ont estimé la variation spatiale de l'abondance relative de différents stocks \citep{thorsonDietAnalysisUsing2022}. Nous avons contraint le modèle à un maximum de 20 et 35 dimensions de base pour les lisses de semaine d'échantillonnage et d'emplacement, ce qui a permis un degré relativement élevé de non-linéarité (pour le contexte, deux dimensions de base approximent une fonction quadratique), tout en assurant la convergence du modèle.

Les prédictions d'un modèle Tweedie multivarié reflètent des estimations continues de la densité moyenne d'une observation (ici un stock donné) dans un échantillon, ainsi que son erreur-type. Les moyennes et les erreurs-types sont ensuite remises à l'échelle comme estimations de composition bornées par 0 et 1 \citep{thorsonDietAnalysisUsing2022, thorsonMultivariateTweedieSelfweightingLikelihood2023}. Nous avons utilisé le modèle ajusté pour générer des prédictions d'effets marginaux représentant a) les prédictions saisonnières moyennes, b) spatiales moyennes, c) saisonnières spatialement-explicites, et d) annuelles spatialement-explicites de la composition des stocks dans la pêcherie au saumon chinook. Ces prédictions ont été utilisées pour évaluer qualitativement les patrons d'abondance spécifiques aux stocks. Nous avons restreint les prédictions saisonnières entre mai et octobre (mai à septembre pour les strates ouest) parce que l'échantillonnage était spatialement déséquilibré à l'extérieur de cette période, ce qui peut résulter en des prédictions de modèle non fiables.

Dans le cas des effets spatiaux moyens, nous présentons des prédictions spatiales mises à l'échelle par l'abondance prédite maximale d'un stock. Les prédictions mises à l'échelle peuvent être interprétées comme représentant la variation spatiale de l'abondance \textit{dans} un stock, tandis que les prédictions spatiales non mises à l'échelle représentent la variation spatiale dans la composition des stocks. Les prédictions spatiales moyennes sont projetées sur une grille, représentant une surface lisse de variabilité dans la composition des stocks. Inversement, les prédictions saisonnières et annuelles spatialement-explicites sont générées pour des emplacements ponctuels spécifiques, pas une grille, pour rendre la visualisation possible. Nous avons généré des prédictions pour un seul emplacement près du centre de chaque strate (symboles X dans la Figure \ref{fig:sampling-map}). Nous notons que ces points ont été choisis arbitrairement et s'ils étaient déplacés, les prédictions changeraient.

Pour évaluer l'ajustement du modèle, nous avons calculé des résidus quantiles randomisés avec des effets fixes à leurs estimations de vraisemblance maximale et des effets aléatoires échantillonnés avec la chaîne de Markov Monte Carlo (CMMC) \citep[\textit{sensu}][]{rufenerBridgingGapCommercial2021}. Nous avons utilisé ces valeurs simulées pour comparer la composition des stocks prédite par le modèle à la composition des stocks observée et assurer que le modèle générait des prédictions plausibles, c.-à-d. vérification prédictive postérieure. Nous avons ajusté les modèles en utilisant le paquet mgcv \citep{woodFastStableRestricted2011}, généré des prédictions en utilisant le paquet mvtweedie \citep{thorsonDietAnalysisUsing2022, thorsonMultivariateTweedieSelfweightingLikelihood2023}, et généré des résidus CMMC en utilisant le paquet sdmTMB \citep{andersonSdmTMBPackageFast2022}. Toutes les analyses ont été conduites dans R version 4.4.0 \citep{rcoreteamLanguageEnvironmentStatistical2021} et les données pertinentes et le code de modèle sont disponibles à \url{https://doi.org/10.5281/zenodo.15843981}.

\subsubsection{Analyse de sélectivité}

Nous avons utilisé le modèle ajusté pour paramétrer une simulation Monte Carlo pour tester l'évidence de sélectivité du régime alimentaire de l'ÉRES. Spécifiquement, nous avons testé si la composition des stocks observée des restes de proies de l'ÉRES différait de la composition des stocks prédite, telle qu'indexée par la pêcherie, tout en tenant compte du nombre relativement petit de restes de proies collectés. Puisque des échantillons de pêcherie concurrents n'ont pas été collectés pendant la période d'échantillonnage précoce des proies, cette analyse a été restreinte aux restes de proies collectés pendant la période actuelle, qui ont tous été collectés dans des emplacements à l'ouest du centre du détroit de Juan de Fuca (Figure \ref{fig:sampling-map}).

Le modèle dépendant de la pêcherie a été utilisé pour prédire un vecteur de composition des stocks moyenne, $\boldsymbol{\hat{\pi}}$, de longueur $S$ (où $S$ est le nombre de stocks) pour chaque événement d'échantillonnage de proies de l'ÉRES basé sur l'emplacement, la semaine, et l'année dans lesquels l'échantillon de proie a été collecté. Puisque les événements d'échantillonnage de l'ÉRES collectaient rarement plus d'un échantillon d'un emplacement donné, nous avons regroupé les échantillons collectés de la même strate, semaine, et année, en utilisant la latitude et longitude moyennes des échantillons originaux, pour créer un nouvel événement d'échantillonnage $j'$ pour l'analyse de sélectivité. Nous avons tenu compte de l'incertitude du modèle en assumant que la contribution proportionnelle de chaque stock provenait d'une distribution normale aléatoire avec moyenne $\hat{\pi_s}$ et écart-type $\sigma_{\pi_s}$, qui ont été remis à l'échelle pour assurer que l'estimation de composition était bornée par 0 et 1. Nous avons ensuite utilisé la composition des stocks observée à $j'$ basée sur l'échantillonnage des restes de proies $\pi$ et $\hat{\pi}$ comme les proportions moyennes dans des tirages aléatoires d'une distribution multinomiale. Nous avons tiré $n_{sim}$ échantillons où $n_{sim}$ est égal au nombre d'échantillons collectés dans un événement d'échantillonnage de l'ÉRES $n_{j'}$, résultant en deux vecteurs $\boldsymbol{Y_{obs_{j'}}}$ et $\boldsymbol{\hat{Y_{sim_{j'}}}}$ représentant les comptes individuels de chaque stock à chaque événement. Nous avons sommé à travers tous les événements d'échantillonnage pour calculer le nombre total d'individus dans chaque stock à travers tous les événements, $\boldsymbol{Y_{obs}}$ et $\boldsymbol{\hat{Y_{sim}}}$. Finalement, nous avons converti les deux vecteurs en proportions et calculé la différence $\boldsymbol{\Delta{Y}}$.

L'analyse de simulation tient explicitement compte de la variabilité saisonnière, spatiale, et interannuelle dans la composition des stocks, tout en propageant l'incertitude due à l'assignation de stock génétique, aux paramètres estimés dans le modèle de pêcherie, et aux tailles d'échantillon des restes de proies eux-mêmes. Nous avons répété l'analyse de simulation 500 fois pour tenir compte de la variabilité dans chaque dimension d'incertitude. Nous avons utilisé la distribution, parmi les simulations, de $\boldsymbol{\Delta{Y}}$ pour visualiser la sélectivité avec des valeurs positives (négatives) représentant un stock se produisant plus (moins) communément dans la composition des stocks observée que prédite.

\subsubsection{Composition de la taille}

Nous avons ajusté un modèle similaire pour quantifier la variation dans la composition de la taille du saumon chinook capturé par la pêcherie récréative. Le modèle était équivalent à l'Équation \ref{eq:comp_eq}, mais au lieu de l'échantillon $j$ représentant les individus appartenant au stock $s$, il représentait les individus appartenant à la classe de taille $c$. Puisque les échantillons individuels de pêcherie récréative ont été mesurés, et pouvaient donc être assignés à un casier de taille avec zéro erreur, nous avons assumé que la contribution de chaque classe suivait une distribution binomiale négative \citep[`NB1', ][]{hilbeNegativeBinomialRegression2011} modélisée avec un lien logarithmique. Comme avec le modèle de composition des stocks, les prédictions moyennes et leurs erreurs-types ont été mises à l'échelle à des estimations proportionnelles post-hoc, qui ont été utilisées pour visualiser les effets marginaux et conduire un exercice de simulation testant l'évidence de sélectivité. L'analyse de sélectivité de taille a propagé l'incertitude associée à la taille estimée des restes de proies due à l'erreur de vieillissement et à la variabilité dans les relations taille-selon-l'âge spécifiques aux stocks (ceci est largement analogue à l'incertitude d'assignation de stock dans la section précédente ; (détails dans l'Appendice \ref{size-at-age})). Nous avons évalué la performance du modèle en utilisant les diagnostics décrits précédemment.

\subsection{Analyse de sensibilité}

Nous avons conduit deux analyses de sensibilité. La première visait à tenir compte des déviations entre la composition des stocks de la pêcherie récréative et les restes de proies de l'ÉRES due à la prédation sélective de taille par les ÉRES. Nous avons exclu les échantillons de la pêcherie récréative plus petits que 75 cm de longueur à la fourche (>70\% des restes de proies estimés être plus grands que cette taille) et réajusté le modèle (Équation \ref{eq:comp_eq}). Dans la seconde, nous avons tenu compte des échantillons de restes de proies qui ont été collectés pendant des moments et emplacements quand les pêcheries étaient fermées ou impactées par des limites de taille (Figure \ref{fig:temporal-fishery-samples-management}). Nous avons exclu le petit nombre d'échantillons impactés (11 des échantillons de stock et huit des échantillons de taille) et répété l'analyse de sélectivité. Des détails additionnels sur les méthodes et résultats sont rapportés dans l'Appendice \ref{sensitivity}.