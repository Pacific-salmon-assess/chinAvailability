Notre compréhension de la relation entre les ÉRES et le saumon quinnat a évolué au cours des cinq dernières décennies. Depuis l'identification de l'écotype « résident » se nourrissant de poissons \citep{fordSelectiveForagingFisheating2006} jusqu'à la corrélation entre les indices d'abondance du saumon quinnat à l'échelle de la côte et la mortalité des épaulards résidents \citep{fordChinookSalmonPredation2010}, l'importance du saumon quinnat comme proie principale est incontestée. Cependant, la variabilité de l'abondance et de la distribution saisonnière des stocks contribuant aux indices de saumon quinnat à l'échelle de la côte, ainsi que l'influence probable d'autres menaces, ont obscurci la relation entre l'abondance du quinnat et la dynamique des populations d'ÉRES \citep{velez-espinoRelativeImportanceChinook2015, nelsonIdentifyingDriversDemographic2024, saygiliPrevalenceChinookSalmon2024}.

Dans cette étude, nous avons exploité des échantillons de restes de proies d'ÉRES et de pêches récréatives, tous deux collectés dans l'habitat essentiel des ÉRES dans le sud de la Colombie-Britannique, pour explorer trois questions interconnectées. Premièrement, quels stocks de saumon quinnat et quelles classes de taille sont les plus couramment observés dans l'alimentation des ÉRES ? Deuxièmement, quelle est la composition du champ de proies disponible pour les ÉRES, particulièrement dans les endroits où les données sur les restes de proies sont limitées ? Troisièmement, y a-t-il des preuves de butinage sélectif par stock ou par taille chez les ÉRES ? Nous présentons nos résultats dans le contexte plus large des changements dans l'abondance du saumon quinnat, de la variabilité du supplément de pisciculture entre les stocks, et des facteurs potentiels de sélectivité par les ÉRES. Enfin, nous relions nos résultats aux actions de gestion potentielles qui pourraient améliorer la disponibilité du saumon quinnat pour les ÉRES. Reconnaître les différences entre les régimes alimentaires observés et le champ de proies disponible fournit un degré de granularité nécessaire pour envisager les opportunités d'options de gestion raffinées à l'appui des besoins en proies des ÉRES.

\subsection{Composition des restes de proies des ÉRES}

Les échantillons de restes de proies d'ÉRES collectés pendant les mois d'été entre le détroit Juan de Fuca et les îles San Juan étaient dominés par un sous-ensemble de stocks de saumon quinnat du fleuve Fraser. Les poissons Fraser Spring $5_2$ et Summer $5_2$, des stocks avec un cycle vital de saumoneau d'un an et une période de montaison précoce, étaient relativement abondants en juin et juillet, tandis que les individus Fraser Summer $4_1$ de saumoneau de l'année constituaient la composante dominante dans l'ensemble, particulièrement en juillet et août. Les stocks restants ont été observés à une abondance relativement faible. Il y avait des preuves que les stocks précoces de saumoneaux d'un an constituaient une proportion plus petite du régime alimentaire dans la période actuelle que dans la période précoce ; cependant, nous n'avons pas pu tester cette hypothèse en raison des différences dans le moment et l'emplacement de l'échantillonnage. Bien que nous n'ayons pas pu identifier de manière fiable le saumon quinnat d'origine de pisciculture dans les restes de proies d'ÉRES, sept échantillons ont été identifiés sur la base de l'ITGP et provenaient de piscicultures du bas Fraser Fall, COIV, et COIV.

Nous avons également évalué la composition de taille des régimes alimentaires des ÉRES, en utilisant les écailles pour estimer la longueur à la fourche moyenne des restes de proies échantillonnés basée sur les relations taille-âge. Le saumon quinnat prédit comme étant plus petit que 65 cm de longueur à la fourche était largement absent des échantillons identifiés par âge, qui étaient plutôt dominés par les classes de taille de 75-85 et >85 cm. Ces tendances sont cohérentes avec les observations estivales antérieures de la mer des Salish qui ont trouvé que les populations intérieures du fleuve Fraser et les classes d'âge plus anciennes étaient des composantes majeures des régimes alimentaires des ÉRES \citep{fordSelectiveForagingFisheating2006, hansonSpeciesStockIdentification2010}.

Trop peu d'échantillons de restes de proies ont été collectés pour tenir compte statistiquement de la variabilité entre les périodes d'échantillonnage, les semaines d'échantillonnage et les emplacements spatiaux. Bien que les observations d'emplacements et de mois avec un plus grand nombre d'échantillons soient plus fiables que celles avec moins, sans modèle nous ne pouvons pas présenter d'estimations d'incertitude fiables ou déterminer la taille d'échantillon minimale nécessaire pour tirer des inférences de manière fiable \citep[p. ex.,][]{tritesDietaryAnalysisFecal2005}. Nous mettons en garde contre l'hypothèse que les régimes alimentaires des ÉRES dans des endroits ou des mois sans échantillons seront similaires aux observations ailleurs, étant donnée la variabilité saisonnière et spatiale dans la composition des stocks et des tailles de saumon quinnat.

\subsection{Composition de la pêche récréative}

Les stocks de saumon quinnat diffèrent dans leur distribution, leurs caractéristiques physiques et leur moment de migration en eau douce \citep{healeyLifeHistoryChinook1991, weitkampMarineDistributionsChinook2010}, résultant en une variabilité spatiale et saisonnière substantielle dans l'abondance spécifique aux stocks dans les habitats marins \citep{sheltonUsingHierarchicalModels2019, freshwaterIntegratedModelSeasonal2021}. Nous avons exploité un grand nombre d'échantillons collectés dans les pêches récréatives canadiennes pour quantifier la variabilité dans la composition des stocks et des tailles de saumon quinnat à des échelles particulièrement fines. Nous avons trouvé des différences marquées dans la composition de la pêche récréative entre les semaines d'échantillonnage à l'intérieur d'un endroit et entre des endroits distants de cinq à dix kilomètres à l'intérieur d'une semaine d'échantillonnage.

Un facteur principal de variation temporelle dans la composition des stocks de saumon quinnat est la contribution relative des stocks résidents, qui sont présents dans les habitats du plateau continental méridional toute l'année, et des stocks migrateurs qui deviennent plus abondants à mesure que les individus matures migrent vers les habitats d'eau douce. Par exemple, le stock de Puget Sound est présent dans la mer des Salish toute l'année \citep{oneillMarineDistributionLife2009, freshwaterIntegratedModelSeasonal2021}, et dominait la composition des pêches récréatives, particulièrement tôt en été. Inversement, les stocks du fleuve Fraser, à l'exception des individus de montaison automnale, migrent typiquement vers le nord le long du plateau continental ou au large \citep{ctc2021AnnualReport2022}. La composition des stocks montrait des pics successifs alors que les reproducteurs de retour Spring $4_2$, puis Spring $5_2$, Summer $5_2$, et Summer $4_1$ migraient vers le fleuve Fraser. Les individus COIV, qui atteignent typiquement la maturité dans les eaux de l'Alaska \citep{sheltonUsingHierarchicalModels2019}, et les individus Fraser River Fall, qui sont résidents le long du plateau et dans le nord du détroit de Georgie, étaient abondants dans des endroits spécifiques à la fin de l'été. Notamment, les individus Columbia River Spring étaient rares, même dans les zones extérieures, suggérant que ce stock peut migrer plus au large jusqu'à atteindre le fleuve Columbia.

Les tendances spatiales différaient également entre les stocks. Les individus COIV étaient largement contraints aux eaux côtières du lac Nitinat à Port Renfrew. Les individus Columbia River Summer/Fall étaient abondants du banc Swiftsure vers l'ouest jusqu'au banc La Perouse. Cette région incluait également une contribution relativement importante de poissons Fraser River Fall, cohérente avec les données récentes de prises accessoires au chalut \citep{lagasseReviewSalmonBycatch2024}, et indiquant qu'elle pourrait être une zone d'élevage pour ce stock. Les individus Fraser River Summer $4_1$ étaient les plus abondants dans le détroit Juan de Fuca. Enfin, les individus de Puget Sound étaient répandus du Swiftsure à l'est jusqu'au sud du détroit de Georgie, mais étaient plus dominants dans les zones côtières.

Les différences dans la composition des tailles sont associées à la composition des stocks parce que le saumon quinnat varie en taille et âge à maturité \citep{healeyLifeHistoryChinook1991}, mais elles reflètent également des différences ontogénétiques alors que les poissons continuent de grandir pendant l'été et montrent des changements dans l'utilisation de l'habitat \citep{freshwaterChinookSalmonDepth2024, freshwaterSeasonalVariabilityCondition2024}. Généralement, les poissons plus petits que 65 cm étaient les plus abondants tôt et tard dans l'année, avant et après que les individus reproducteurs aient migré à travers la zone. Les classes de taille 75-85 et >85 cm montraient la tendance opposée, tandis que la classe de taille 65-75 cm montrait une faible variabilité saisonnière. Les deux classes de taille plus grandes sont principalement des poissons matures, mais cette dernière inclut à la fois des poissons sexuellement matures, ainsi que des individus qui resteront en mer pour une année supplémentaire \citep{freshwaterChinookSalmonDepth2024, freshwaterSeasonalVariabilityCondition2024}. Les tendances spatiales soutenaient également cette hypothèse---les poissons plus petits que 75 cm étaient répandus et particulièrement abondants au large du banc Swiftsure et au sud de Victoria, des endroits communément considérés comme habitat d'élevage pour les poissons résidents, tandis que ceux plus grands que 75 cm étaient concentrés près de l'embouchure du détroit Juan de Fuca. Les points chauds d'abondance pour les classes de taille plus grandes étaient similaires aux endroits où les comportements de recherche de nourriture des ÉRES sont le plus couramment observés \citep{thorntonSouthernResidentKiller2022}.

Les estimations de composition des stocks et des tailles ont le potentiel d'être fortement influencées par les mesures de gestion telles que les pêches sélectives par marque et les limites de taille. Bien qu'il n'ait pas été possible de tenir compte statistiquement de ces influences, nos résultats ne correspondent pas uniformément au moment et à l'emplacement des interventions de gestion. Par exemple, des changements graduels dans la composition des stocks et des tailles tout au long de l'été, plutôt que des transitions abruptes associées à un changement de régime de pêche, étaient présents dans des strates telles que Sooke/Victoria qui ont eu des limites de taille maximale en place pendant l'été pour toute la série chronologique. Le banc Swiftsure, Sooke/Victoria, et les îles Gulf du sud montraient des tendances similaires dans la composition des tailles les unes aux autres malgré des régimes de gestion très différents. Enfin, Nitinat avait une distribution de taille plus petite que Port Renfrew malgré des interventions de gestion beaucoup moins restrictives. Ainsi, bien que l'impact des interventions de gestion sur les estimations de composition ne soit probablement pas nul, il est aussi improbable qu'elles soient uniquement responsables de la variation saisonnière et spatiale. Nous explorons ces facteurs plus en détail dans la section Limitations.

\subsection{Abondance relative des poissons d'origine de pisciculture}

Fournir des estimations précises de l'abondance relative des poissons d'origine de pisciculture dans les restes de proies d'ÉRES et l'habitat essentiel est difficile. Les données des restes de proies eux-mêmes ne peuvent être assignées comme d'origine de pisciculture qu'avec une certitude relativement élevée basée sur l'ITGP. Pourtant, les estimations des contributions de pisciculture au régime alimentaire basées uniquement sur l'ITGP seront biaisées vers le bas parce que les assignations ITGP ne sont pas disponibles pour les stocks américains et la couverture ITGP variait entre les années et entre les piscicultures pour les stocks canadiens alors que la méthode était graduellement introduite.

La précision des estimations de contribution de pisciculture pour les échantillons dépendants de la pêche est plus élevée parce que les individus de pisciculture peuvent être identifiés sur la base de nageoires adipeuses manquantes. Puisque les taux de coupure de la nageoire adipeuse sont relativement élevés dans les piscicultures de Washington et de l'Oregon \citep{andersonReviewHatcheryReform2020, wdfwAnadromousSalmonSteelhead2021, odfwFishPropagationAnnual2022}, les poissons d'origine américaine pourraient normalement être assignés de manière fiable. Les taux de marquage canadiens, cependant, sont beaucoup plus bas et les poissons canadiens non marqués qui appartenaient à une classe d'âge avec un faible taux d'ITGP ne pouvaient pas être identifiés de manière fiable comme étant d'origine de pisciculture ou sauvage. Le Programme de mise en valeur des salmonidés (le programme de MPO responsable des opérations de pisciculture) quantifie une dimension des impacts de pisciculture en estimant la proportion d'individus d'origine de pisciculture sur les frayères (pHOS). Pourtant, seulement un sous-ensemble de populations à l'intérieur de chaque stock sont évaluées et les estimations pHOS ne reflètent pas l'abondance de pisciculture dans les retours aux installations de mise en valeur ou dans la récolte. Ainsi, il n'est pas possible actuellement de dériver un indice de l'abondance relative des poissons de pisciculture à l'intérieur d'un stock canadien donné, particulièrement pendant la résidence marine où plusieurs cohortes co-occurent.

\subsection{Comportement de recherche de nourriture des ÉRES}

Évaluer la disponibilité des proies pour les ÉRES exige une compréhension nuancée des besoins de recherche de nourriture et du comportement de recherche de nourriture des ÉRES. Les populations d'épaulards exhibent une spécialisation écologique rigoureuse \citep{rieschCulturalTraditionsEvolution2012}, qui réduit la compétition \citep{whiteheadConsequencesCulturallydrivenEcological2018}. Chez les épaulards, cette focalisation sur une petite portion du spectre de ressources soutient la coexistence de trois écotypes dans les eaux canadiennes du Pacifique (épaulards résidents, transitoires et du large). Bien que la spécialisation comportementale réduise la compétition et favorise la survie quand les proies ne sont pas limitantes, la sélectivité de proies hautement spécialisée peut être néfaste à la survie d'une population quand les populations de proies principales sont moins abondantes, comme le démontre la population d'ÉRES.

Bien que la spécificité culturelle de la sélection de proies des ÉRES semble être rigide, la population montre des changements dans le comportement de recherche de nourriture et la distribution spatiale qui sont cohérents avec une réponse à la disponibilité des proies. Les données de régime alimentaire hivernal indiquent une diversité d'espèces de proies relativement élevée, probablement en réponse aux densités saisonnières plus faibles de saumon quinnat \citep{hansonEndangeredPredatorsEndangered2021} ; cependant, le saumon quinnat demeure une composante importante du régime alimentaire tout au long de l'année \citep{vanciseSpatialSeasonalForaging2024}. En concert avec la diversité accrue dans la sélection de proies hivernales, les mouvements des ÉRES sont également plus irréguliers et s'étendent dans toute leur aire de répartition \citep{hansonAssessingMovementsOccurrence2017}. La fidélité saisonnière au site semble être corrélée avec l'augmentation de l'abondance du saumon quinnat, commençant à la fin mars avec une utilisation accrue des habitats près de l'embouchure du fleuve Columbia \citep{zamonWinterObservationsSouthern2007, hansonAssessingMovementsOccurrence2017}. Au début juin, les ÉRES sont plus communs dans les environs du banc Swiftsure \citep{thorntonSouthernResidentKiller2022}, et leur utilisation des habitats du sud de la C.-B. continue de croître tout au long de l'été \citep{ettingerShiftingPhenologyEndangered2022, stewartTraditionalSummerHabitat2023}. Crucialement, la condition physique des ÉRES covarie avec les mouvements saisonniers et est typiquement la plus mauvaise au printemps et au début de l'été, puis se rétablit pendant la fin de l'été et l'automne (Fearnbach et al. 2020; J. Durban, comm. pers.). Les tendances saisonnières de la condition physique sont probablement associées à une réduction de la contribution du saumon quinnat aux régimes alimentaires des ÉRES et à une dépense énergétique accrue pour se nourrir sur une base de proies plus diverse et moins spatialement concentrée. Nous notons que ces tendances reflètent le comportement des ÉRES au cours des décennies récentes, qui peut avoir divergé des tendances historiques en raison de grandes réductions de la taille de la population d'ÉRES, ainsi que des changements dans l'abondance relative de stocks spécifiques de saumon quinnat (p. ex., les populations de saumoneaux d'un an du fleuve Fraser).

\subsection{Sélectivité des ÉRES}

Bien que la co-occurrence du prédateur et de la proie soit un prérequis de la recherche de nourriture, les prédateurs peuvent sélectionner les proies de manière disproportionnée à leur abondance. Nous avons trouvé de larges similitudes entre la composition des stocks des restes de proies d'ÉRES et les données dépendantes de la pêche, incluant un déclin saisonnier dans les stocks Fraser River Spring $5_2$ et Summer $5_2$, une augmentation saisonnière dans le stock Fraser River Summer $4_1$, et la présence persistante du stock de Puget Sound. Pourtant, la variabilité temporelle et spatiale dans la composition des stocks et des tailles de saumon quinnat interagit pour produire un champ de proies variable pour les ÉRES, même aux échelles relativement petites sur lesquelles les restes de proies ont été collectés ici. Pour tenir compte de la variabilité spatio-temporelle dans la composition du saumon quinnat, ainsi que de l'échantillonnage inégal des restes de proies d'ÉRES, nous avons utilisé un modèle de simulation pour tester s'il y avait des preuves que des stocks et des classes de taille spécifiques étaient présents dans les restes de proies de manière disproportionnée à leur abondance telle qu'indexée par la pêche.

Les ÉRES montraient une sélection positive pour les stocks Fraser Spring $5_2$, Summer $5_2$, et Summer $4_1$, ainsi qu'une sélection négative vers les stocks COIV, Columbia Summer/Fall, Puget Sound, et « Autres ». Les différences entre les restes de proies et les échantillons dépendants de la pêche étaient négligeables pour les stocks restants après propagation de l'incertitude d'échantillonnage, d'identification de stock et de modèle. De plus, les ÉRES montraient un gradient de sélectivité de taille. Les classes de taille les plus petites et les plus grandes étaient sous- et sur-représentées, respectivement, dans les régimes alimentaires par rapport à la pêche récréative. La sélectivité pour les classes de taille intermédiaires était similaire directionnellement, mais plus incertaine.

Nos résultats sont cohérents avec des preuves que les ÉRES ciblent préférentiellement les saumons quinnat plus grands et plus âgés comme proies, présumément pour maximiser les retours énergétiques par élément de proie \citep{fordSelectiveForagingFisheating2006, oneillEnergyContentPacific2014}. Cependant, plusieurs stocks avec la plus grande taille corporelle (Fraser Spring $5_2$, Fraser Summer $5_2$, Fraser Summer $4_1$, et COIV) montraient des preuves à la fois de sélectivité positive et négative dans les restes de proies d'ÉRES. La sélectivité de stock était également encore présente après que nous ayons contraint notre analyse aux classes de taille de saumon quinnat les plus grandes (Figure \ref{fig:comb-sel-stock}).

Notre première hypothèse est que la variabilité dans le contenu lipidique, ainsi que la taille corporelle, influence la valeur de proie. Typiquement, le saumon quinnat subissant des migrations de frai plus longues vers les bassins versants intérieurs ont un contenu lipidique plus élevé, par gramme de poids corporel, que les populations côtières migrant sur des distances plus courtes \citep{quinnBehaviourEcologyPacific2018}. Les différences de contenu lipidique entre stocks sont présentes dans les habitats marins et d'eau douce terminaux une fois que les migrations de frai ont commencé \citep{oneillEnergyContentPacific2014, lernerSeasonalVariationLipid2023}, mais aussi dans les zones marines non terminales où à la fois le saumon quinnat et les ÉRES recherchent activement de la nourriture \citep{hendriksBehaviourMovementReturn2024, freshwaterSeasonalVariabilityCondition2024}. Dans ces zones marines, les stocks Fraser River Spring $5_2$ et Summer $5_2$ ont un contenu lipidique environ 10 % plus élevé (par unité de masse corporelle) que les stocks Fraser River Summer $4_1$ ou Upper Columbia River et Snake River Summer/Fall, 20 % plus élevé que les stocks Fraser River Fall, Puget Sound, ou lower Columbia River Fall, et 30 % plus élevé que le stock COIV \citep{freshwaterSeasonalVariabilityCondition2024}. Généralement, les stocks avec des preuves de sélectivité positive dans les restes de proies d'ÉRES étaient ceux qui migrent vers les bassins versants intérieurs, tandis que ceux avec une sélection négative sont plus susceptibles d'inclure des populations de saumoneaux de l'année qui migrent sur des distances plus courtes vers des sites de frai côtiers. Il y avait une sélectivité neutre vers plusieurs stocks de montaison printanière avec des populations de frai intérieures, ce qui pourrait être dû à leur taille moyenne plus petite (Fraser River Spring $4_2$) ou à une abondance proportionnelle très faible résultant en un pouvoir statistique limité.

Bien que la variation dans le contenu lipidique soit une explication parcimonieuse, il est possible que la sélectivité de stock soit influencée par d'autres facteurs. La sélection d'un saumon quinnat individuel par les ÉRES peut être influencée par l'emplacement du poisson dans la colonne d'eau \citep{wrightFinescaleForagingMovements2017}, l'environnement de bruit ambiant \citep{tennessenMalesMissFemales2024}, et les caractéristiques bathymétriques qui interfèrent avec la réception d'un signal d'écholocation (p. ex., composition du substrat, rugosité). Ces facteurs peuvent réduire l'accessibilité des proies par une efficacité diminuée de l'écholocation ou en réduisant la probabilité d'une poursuite réussie.

Les différences entre les saumons quinnat dans l'utilisation d'habitats qui varient dans ces traits peuvent déterminer leur disponibilité comme proies. Par exemple, les distributions verticales diffèrent entre les stocks de saumon quinnat en raison de voies migratoires uniques ou de différences dans la proportion relative d'individus matures et immatures \citep{freshwaterChinookSalmonDepth2024}. Le saumon quinnat COIV dans les zones terminales est orienté vers le fond et peut être moins susceptible d'exhiber des comportements de recherche de nourriture qui les exposent à la prédation par les ÉRES \citep[p. ex.,][]{waltersRecruitmentLimitationConsequence1993}. D'autre part, les poissons liés au fleuve Fraser semblent migrer plus rapidement à travers la mer des Salish que les individus de Puget Sound \citep{hendriksBehaviourMovementReturn2024}, ce qui intuitivement pourrait réduire leur disponibilité comme proies---l'opposé de nos résultats de sélectivité. Bien que notre simulation ait tenu compte des différences entre stocks dans la distribution spatiale, elle n'est pas appropriée pour évaluer les tendances granulaires dans l'utilisation de l'habitat. La recherche future visant à comprendre le comportement et la physiologie spécifiques aux stocks du saumon quinnat peut être nécessaire pour comprendre les tendances de sélectivité des ÉRES.

Enfin, nos résultats peuvent être influencés par la sélectivité de la pêche indépendamment de la sélectivité des ÉRES. Nous évaluons cette possibilité dans la section Limitations et orientations de recherche future.

\subsection{Implications pour la gestion}

La valeur de différents stocks de saumon quinnat comme proies pour les ÉRES peut être influencée par au moins trois caractéristiques : la condition physique plus faible des ÉRES pendant le printemps, la qualité nutritionnelle de différents stocks de saumon quinnat, et le chevauchement spatio-temporel entre le saumon quinnat et les habitats de recherche de nourriture des ÉRES. Étant donné ces hypothèses, ainsi que les tendances divergentes dans l'abondance entre les stocks de saumon quinnat, nous suggérons que la disponibilité des proies pour les ÉRES dans l'habitat essentiel canadien pourrait être maximisée en atteignant deux objectifs---augmenter l'abondance des stocks à risque Fraser River Spring $5_2$ et Summer $5_2$ et augmenter la taille corporelle ou la densité énergétique des stocks de saumoneaux de l'année relativement abondants qui sont présents dans les habitats de recherche de nourriture des ÉRES.

Les individus Fraser River Spring $5_2$ et Summer $5_2$ peuvent être des éléments de proie particulièrement importants pendant la période du printemps et du début de l'été parce que leur migration se produit après les populations Columbia River Spring, mais avant celle des stocks Columbia River Summer/Fall, Fraser River Summer $4_1$, et Fraser River Fall relativement plus abondants \citep{keeferStockspecificMigrationTiming2004, parkenGeneticCodedWire2008, jepsonPopulationCompositionMigration2010}. Les individus Fraser River Spring $5_2$ et Summer $5_2$ étaient présents dans les restes de proies d'ÉRES de manière disproportionnée à leur abondance, cohérent avec des preuves antérieures que les individus de ces stocks sont des proies communes \citep{hansonSpeciesStockIdentification2010}. Les deux stocks sont également riches en lipides et de grande taille corporelle (Xu et al. 2020, Lerner and Hunt 2023, Freshwater and King 2024, cette étude). Enfin, l'abondance des populations Fraser River Spring $5_2$ et Summer $5_2$ a décliné par rapport aux niveaux historiques \citep{cosewicChinookSalmonOncorhynchus2018, dfoRecoveryPotentialAssessment2020a, dfoRecoveryPotentialAssessment2021} et d'autres stocks plus abondants présents dans l'habitat essentiel tôt dans l'année sont typiquement plus petits et ont un contenu lipidique moyen plus faible \citep{oneillEnergyContentPacific2014, freshwaterIntegratedModelSeasonal2021, hendriksBehaviourMovementReturn2024, freshwaterSeasonalVariabilityCondition2024}.

Pourtant, il y a des options limitées pour augmenter rapidement l'abondance des stocks Fraser River Spring $5_2$ et Summer $5_2$. Les taux d'exploitation marine canadiens ont été faibles pendant plusieurs années \citep{dfoFraserChinookFishery2023}. Aucune population Spring $5_2$ ou Summer $5_2$ n'a actuellement de programmes de mise en valeur à grande échelle (définis ici comme plus de 140 000 relâches de juvéniles par année (Programme de mise en valeur des salmonidés de MPO, données non publiées)). Il est bénéfique de continuer des efforts diversifiés pour reconstruire ces stocks via la restauration d'habitat, la pression de récolte réduite, et la mise en valeur ciblée via des piscicultures de conservation à petite échelle même s'ils sont peu susceptibles d'augmenter l'abondance à court terme.

Bien que de nombreux stocks de saumon quinnat saumoneaux d'un an aient décliné en abondance, l'abondance de plusieurs stocks de saumoneaux de l'année qui sont communs dans l'habitat essentiel des ÉRES a augmenté ou est demeurée stable (Atlas et al. 2023, CTC 2024, cette étude)\nocite{atlasTrendsChinookSalmon2023,ctcAnnualReportCatch2024a}. Par exemple, les retours Fraser River Summer $4_1$ en 2023 étaient les plus importants jamais enregistrés \citep{ctcAnnualReportCatch2024a}, continuant une augmentation de plusieurs décennies en abondance. Ce stock a également un contenu lipidique relativement élevé et était sélectionné de manière disproportionnée par les ÉRES, suggérant qu'il est un élément de proie important ; cependant, ce stock n'est disponible dans la zone d'étude que relativement tard en été en raison de leur distribution nordique \citep{weitkampMarineDistributionsChinook2010, freshwaterIntegratedModelSeasonal2021, ctc2021AnnualReport2022}. Inversement, les individus Columbia Fall, Fraser Fall, et Puget Sound sont disponibles dans la zone d'étude sur une période plus longue en raison de leurs cycles vitaux résidents \citep{weitkampMarineDistributionsChinook2010, sheltonUsingHierarchicalModels2019, freshwaterIntegratedModelSeasonal2021}, mais ont typiquement un contenu lipidique plus faible et, tôt dans l'année, peuvent être plus petits \citep{oneillEnergyContentPacific2014, lernerSeasonalVariationLipid2023, freshwaterSeasonalVariabilityCondition2024}. De plus, plusieurs des populations à l'intérieur de ces stocks, mais pas les populations Fraser River Summer $4_1$, sont fortement mises en valeur par les piscicultures. Les efforts pour augmenter la disponibilité des proies via la mise en valeur devraient tenir compte des différences entre les stocks de saumon quinnat dans les distributions marines et la sélectivité des ÉRES. Par exemple, augmenter l'abondance de l'ECVI/SOMN peut être moins efficace à court terme parce que ces individus sont rarement rencontrés dans la portion occidentale de la zone d'étude et d'autres eaux extérieures où les ÉRES passent une plus grande portion de l'année \citep{stewartTraditionalSummerHabitat2023}. Augmenter l'abondance du stock COIV peut également être inefficace si les ÉRES montrent une sélection négative vers ce stock malgré la co-occurrence spatio-temporelle.

Les changements dans les stratégies de récolte et de mise en valeur peuvent fournir une opportunité d'améliorer les conditions de recherche de nourriture des ÉRES via des changements dans la qualité des proies, plutôt que l'abondance des proies. Par exemple, MPO a incorporé des limites de taille maximale dans plusieurs pêches marines \citep{dfoPacificRegionFinal2023}. L'objectif principal des limites de taille actuelles est de minimiser les impacts de récolte sur les UC à risque ; cependant, ces restrictions augmentent probablement aussi l'abondance locale de grands individus d'autres stocks. Il peut être possible d'augmenter la taille moyenne ou le contenu lipidique d'individus de saumon quinnat saumoneaux de l'année abondants via des changements dans les pratiques de pisciculture. L'âge à maturité est à la fois héréditaire \citep{mckinneyYchromosomeHaplotypesAre2020} et influencé par des facteurs environnementaux pendant l'élevage en eau douce \citep{larsenGrowthModulationAlters2006}. La capacité d'augmenter le contenu lipidique moyen via des changements aux stratégies de mise en valeur est moins claire, mais les tendances divergentes entre stocks suggèrent que c'est un trait héréditaire. Néanmoins, les conséquences écologiques de la mise en valeur sont complexes et les risques associés aux programmes expérimentaux devraient être considérés soigneusement. Par exemple, augmenter l'abondance d'espèces de proies riches en lipides peut bénéficier à d'autres prédateurs de saumons, augmentant les pressions compétitives sur les ÉRES et annulant tout bénéfice anticipé.

Ultimement, des stratégies multiples et diverses peuvent être plus réussies que d'implémenter des changements drastiques dans les réglementations de récolte ou la mise en valeur seule. L'augmentation de la disponibilité des proies via des changements dans la taille et la densité énergétique du saumon quinnat devrait être considérée aux côtés de réductions du bruit des navires pour améliorer le succès de recherche de nourriture des ÉRES et la restauration d'habitat en eau douce de saumon pour améliorer la productivité. Implémenter une gamme d'interventions destinées à reconstruire les populations de saumon quinnat appauvries tout en atténuant simultanément les menaces non-proies et en continuant la recherche sur les mécanismes de déclin dans la productivité du saumon quinnat est cohérent avec des recommandations antérieures \citep{hilbornEffectsSalmonFisheries2012}.

\subsection{Limitations et orientations de recherche future}

Bien que nous ayons incorporé plusieurs jeux de données pour estimer la variabilité dans la préférence de proies des ÉRES entre les saumons quinnat, notre étude avait plusieurs limitations. Nous avons été forcés de faire l'hypothèse que les échantillons dépendants de la pêche reflétaient le saumon quinnat disponible pour les ÉRES parce qu'il n'y a pas de données indépendantes de la pêche disponibles pour quantifier la composition des stocks dans les zones marines. La sélectivité des engins ou le comportement des pêcheurs peuvent causer que les données dépendantes de la pêche diffèrent de la population sous-jacente et ces biais peuvent varier entre les groupes de récolteurs (c.-à-d., les flottilles). Malgré l'inclusion d'au moins trois flottilles potentielles, nous n'avons pas pu estimer ces effets statistiquement.

Les interventions de gestion destinées à minimiser les impacts de récolte sur les UC de saumon quinnat à risque peuvent amplifier les biais associés aux données dépendantes de la pêche. Les restrictions domestiques sur la récolte basées sur la taille et le statut de marque, ainsi que les fermetures de temps-zone, ont varié saisonnièrement, spatialement, et interannuellement, mais ont été présentes dans une certaine portion de l'habitat essentiel canadien depuis 2008 (\citealt{dobsonTechnicalReviewManagement2020} fournissent un résumé détaillé des actions avant 2019, tandis que les fermetures récentes sont résumées dans \citealt{dfoPacificRegionFinal2023}). Généralement, les restrictions étaient les plus sévères dans les zones statistiques à l'est de 20-3 (approximativement le centre du détroit Juan de Fuca), avant le 1er août, et à partir de 2019. Les données collectées à Nitinat, ainsi que dans les strates du banc Swiftsure et de Port Renfrew avant 2019 ou après le 1er août depuis 2019, sont relativement non biaisées parce que seulement les limites de taille et de sac standards étaient en place (Figure \ref{fig:temporal-fishery-samples-management}). De plus, les échantillons de poissons relâchés étaient relativement communs (25-50 % des échantillons par événement depuis 2018) dans les endroits du détroit de Haro aux îles Gulf du sud, ce qui réduira le biais dans les strates des îles Gulf du sud. Ainsi, les estimations de composition prédites sont les plus susceptibles d'être biaisées dans les endroits entre Victoria et Sooke, où les interventions de gestion étaient présentes pendant mai et juin depuis avant 2014 et l'échantillonnage de poissons relâchés était rare.

La combinaison de pêches sélectives par marque avec des limites de taille maximale et un échantillonnage limité de poissons relâchés entre Sooke et Victoria peut avoir réduit la proportion relative d'individus des plus grands groupes de taille et augmenté l'abondance apparente de poissons de Puget Sound, qui sont à la fois abondants et ont un taux de marquage élevé. Pourtant, sans échantillonnage indépendant de la pêche supplémentaire ou des modèles complexes de reconstruction de montaison, il n'est pas possible de quantifier le biais des estimations de composition des stocks et des tailles dans ces zones. Bien que les estimations de composition basées sur les échantillons de pêche puissent être biaisées par les changements dans le régime de gestion, ces effets n'ont pas impacté les résultats de notre analyse de sélectivité. Presque tous les échantillons de restes de proies d'ÉRES de la période d'échantillonnage actuelle ont été collectés dans des endroits à l'ouest de la zone 20-3 et à partir de juillet---c.-à-d., l'analyse de simulation a comparé les restes de proies aux sorties de modèle qui étaient paramétrées avec des données dépendantes de la pêche collectées pendant des temps et endroits avec des mesures de gestion supplémentaires négligeables. Exclure le petit nombre d'échantillons de restes de proies qui ont été collectés quand les pêches récréatives étaient fermées ou des limites de taille étaient en place n'a pas impacté les résultats de l'analyse de sélectivité (Annexe \ref{sensitivity}).

Les restes de proies n'étaient également disponibles que pour une seule espèce, pendant les mois d'été, et sur une portion relativement petite de l'aire de répartition des ÉRES. Nos résultats ne fournissent aucune information sur la sélectivité des ÉRES dans des endroits ou des périodes où les restes de proies n'ont pas été collectés. Les stocks de saumon quinnat d'origine de Columbia peuvent être favorisés dans des endroits tels que la côte de Washington \citep{sheltonUsingHierarchicalModels2019} ou les stocks canadiens avec un moment de montaison plus tardif ou une abondance relative plus grande (p. ex., Fraser River Fall) peuvent être préférés dans le sud du détroit de Georgie \citep{freshwaterIntegratedModelSeasonal2021}. En effet, le saumon quinnat du fleuve Columbia est couramment observé dans les restes de proies d'ÉRES collectés de la côte de Washington, particulièrement pendant l'hiver et au début du printemps \citep{hansonEndangeredPredatorsEndangered2021}. Idéalement, des tests similaires pour la sélectivité pourraient être répliqués dans d'autres endroits et saisons. Il y a aussi des preuves considérables que d'autres espèces de proies peuvent être des composantes importantes du régime alimentaire des ÉRES en dehors des mois d'été \citep{hansonEndangeredPredatorsEndangered2021, vanciseSpatialSeasonalForaging2024}. La recherche devrait être étendue au-delà du saumon quinnat pour mieux comprendre la disponibilité des proies pendant les périodes saisonnières où les ÉRES peuvent être les plus limités en proies. Une telle recherche est particulièrement critique étant donné les déclins récemment observés dans l'abondance du saumon kéta du sud de la C.-B., une autre espèce de proie dominante dans les régimes alimentaires des ÉRES \citep{hansonEndangeredPredatorsEndangered2021, vanciseSpatialSeasonalForaging2024}.

La variabilité entre les stocks et classes de taille de saumon quinnat dans les régimes alimentaires des ÉRES est une fonction à la fois de la sélectivité et de l'abondance locale. En raison de comportements ou distributions spécifiques aux stocks, l'abondance locale dans l'habitat de recherche de nourriture des ÉRES peut ou ne pas être hautement corrélée avec l'abondance absolue d'un stock. Puisque nous n'avons pas pu évaluer l'abondance locale et la sélectivité dans tous les habitats de recherche de nourriture des ÉRES, nous ne pouvons pas fournir une évaluation de l'importance absolue de chaque stock de saumon quinnat comme proie des ÉRES. Plutôt, nos résultats soulignent que le champ de proies disponible pour les ÉRES varie sur de petites échelles spatio-temporelles et notre modèle fournit un outil pour quantifier l'abondance \textit{relative} spécifique aux stocks à des temps et endroits spécifiques. La recherche supplémentaire est nécessaire pour intégrer l'information sur l'abondance spécifique aux stocks et les distributions à échelle fine, ainsi que l'utilisation de l'habitat par les épaulards résidents. La recherche future focalisée sur l'identification des mécanismes contribuant aux dynamiques divergentes entre les stocks de saumon quinnat (p. ex., Fraser River Spring et Summer $5_2$ vs. Summer $4_1$) peut également être précieuse pour informer les efforts de reconstruction des stocks.

Nous avions un pouvoir statistique insuffisant pour modéliser adéquatement la variabilité dans les restes de proies d'ÉRES et n'avons donc pas pu évaluer les changements dans le régime alimentaire entre la période d'échantillonnage précoce et actuelle. Similairement, un manque d'échantillonnage génétique dans les pêches avant 2014 a empêché une analyse des tendances à long terme dans la composition des stocks. En l'absence de ces données, l'abondance de retour spécifique aux stocks facilement disponible (c.-à-d., prise plus échappement) pourrait être utilisée comme proxy pour les tendances à long terme dans l'abondance du saumon quinnat \citep[p. ex.,][]{atlasTrendsChinookSalmon2023}, ainsi que l'abondance dans les zones de recherche de nourriture \citep[p. ex.,][]{wardQuantifyingEffectsPrey2009, nelsonIdentifyingDriversDemographic2024}. Nous notons qu'il n'y a pas de preuves d'un déclin répandu et uniforme dans l'abondance terminale agrégée du saumon quinnat dans la zone d'étude, mais plutôt des tendances divergentes entre et à l'intérieur des stocks (Atlas et al. 2023, CTC 2024, cette étude).

Les données d'abondance terminale fournissent un indice d'abondance minimale. Elles ne peuvent pas tenir compte de l'échappement vers chaque système à l'intérieur d'une région, mais plus important elles ne tiennent pas compte de la récolte marine ou de l'abondance d'individus immatures. Les changements dans le régime de gestion sont moins susceptibles d'impacter les liens entre l'abondance terminale et la disponibilité des proies parce que les pêches commerciales à grande échelle se produisaient historiquement au large de l'habitat essentiel canadien des ÉRES (c.-à-d., côte ouest de l'île de Vancouver, nord de la Colombie-Britannique, et sud-est de l'Alaska). Les estimations d'abondance terminale échouent également à tenir compte des distributions spatio-temporelles spécifiques aux stocks qui impactent leur disponibilité comme proies pour les ÉRES. Nous suggérons que les analyses futures se concentrent sur les stocks qui sont les plus susceptibles de servir de proies, basées sur le chevauchement spatio-temporel avec les zones de recherche de nourriture des ÉRES, plutôt que sur de grands agrégats multi-stocks. Les analyses peuvent être améliorées en mettant à l'échelle les estimations d'abondance par la proportion d'un stock qui migre à travers l'habitat des ÉRES et son temps de résidence. Même si ces améliorations analytiques sont faites, cependant, nous exhortons les chercheurs à être prudents quand ils interprètent les relations statistiques entre l'abondance du saumon quinnat et les dynamiques des ÉRES comme strictement mécanistes, particulièrement étant donné les séries chronologiques relativement courtes d'abondance de saumon qui sont disponibles.

\subsection{Conclusions}

Nous avons utilisé des données dépendantes de la pêche et des échantillons de restes de proies d'ÉRES pour mieux comprendre la variabilité saisonnière et spatiale dans l'abondance du saumon quinnat dans le contexte des proies disponibles pour les ÉRES. Des données suffisantes étaient disponibles pour estimer la composition moyenne des stocks et des tailles dans toute la zone d'étude (banc Swiftsure aux îles Gulf du sud) entre mai et octobre (juin et septembre dans les zones occidentales). Nos résultats soulignent que les ÉRES rencontrent une gamme diverse de stocks de saumon quinnat. Une disponibilité adéquate des proies dépend probablement de l'accès à plusieurs stocks avec des distributions de frai et des cycles vitaux distincts. Il peut être possible d'améliorer l'efficacité des mesures de gestion en concentrant les interventions sur les stocks de saumon quinnat qui sont abondants dans les habitats de recherche de nourriture des ÉRES pendant les périodes de stress énergétique, plutôt que sur ceux qui sont abondants dans les eaux du sud de la Colombie-Britannique dans l'ensemble. Les ÉRES montrent des preuves de sélectivité de taille et de lipides, et semblent être en condition physique particulièrement mauvaise tôt dans l'année. Ainsi, nous suggérons que les actions de gestion destinées à augmenter la disponibilité des proies devraient prioriser l'augmentation de l'abondance de proies grandes et riches en lipides, particulièrement pendant le printemps et le début de l'été, et nous mettons en garde contre l'utilisation d'indices grossiers et multi-stocks d'abondance pour tirer des inférences sur la disponibilité des proies. Enfin, nous notons que nos conclusions reflètent les processus écologiques actuels, qui peuvent être impactés par des changements dans l'abondance et le comportement des ÉRES ou du saumon quinnat, particulièrement étant donné le changement climatique en cours.